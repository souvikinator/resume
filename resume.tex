%-------------------------
% Resume in Latex
% Author : Jake Gutierrez
% Based off of: https://github.com/sb2nov/resume
% License : MIT
%------------------------

\documentclass[letterpaper,11pt]{article}

\usepackage{latexsym}
\usepackage[empty]{fullpage}
\usepackage{titlesec}
\usepackage{marvosym}
\usepackage[usenames,dvipsnames]{color}
\usepackage{verbatim}
\usepackage{enumitem}
\usepackage[hidelinks,colorlinks]{hyperref}
\usepackage{fancyhdr}
\usepackage[english]{babel}
\usepackage{tabularx}
\usepackage{xcolor}
\input{glyphtounicode}


%----------FONT OPTIONS----------
% sans-serif
% \usepackage[sfdefault]{FiraSans}
% \usepackage[sfdefault]{roboto}
% \usepackage[sfdefault]{noto-sans}
% \usepackage[default]{sourcesanspro}

% serif
% \usepackage{CormorantGaramond}
\usepackage{charter}


\pagestyle{fancy}
\fancyhf{} % clear all header and footer fields
\fancyfoot{}
\renewcommand{\headrulewidth}{0pt}
\renewcommand{\footrulewidth}{0pt}

% Adjust margins
\addtolength{\oddsidemargin}{-0.7in}
\addtolength{\evensidemargin}{-0.7in}
\addtolength{\textwidth}{1in}
\addtolength{\topmargin}{-.7in}
\addtolength{\textheight}{1.0in}

\urlstyle{same}

\raggedbottom
\raggedright
\setlength{\tabcolsep}{0in}

% Sections formatting
\titleformat{\section}{
  \vspace{-4pt}\scshape\raggedright\large
}{}{0em}{}[\color{black}\titlerule \vspace{-5pt}]

% Ensure that generate pdf is machine readable/ATS parsable
\pdfgentounicode=1

%-------------------------
% Custom commands
\newcommand{\resumeItem}[1]{
  \item\small{
    {#1 \vspace{-2pt}}
  }
}
\hypersetup{%
     colorlinks=true,
     linkcolor=blue,
     filecolor=blue,
     citecolor = black,      
     urlcolor=teal,
}

\newcommand{\resumeSubheading}[4]{
  \vspace{-2pt}\item
    \begin{tabular*}{0.97\textwidth}[t]{l@{\extracolsep{\fill}}r}
      \textbf{#1} & #2 \\
      \textit{\small#3} & \textit{\small #4} \\
    \end{tabular*}\vspace{-7pt}
}

\newcommand{\resumeSubSubheading}[2]{
    \item
    \begin{tabular*}{0.97\textwidth}{l@{\extracolsep{\fill}}r}
      \textit{\small#1} & \textit{\small #2} \\
    \end{tabular*}\vspace{-7pt}
}

\newcommand{\resumeProjectHeading}[2]{
    \item
    \begin{tabular*}{0.97\textwidth}{l@{\extracolsep{\fill}}r}
      \small#1 & #2 \\
    \end{tabular*}\vspace{-7pt}
}

\newcommand{\resumeSubItem}[1]{\resumeItem{#1}\vspace{-4pt}}

\renewcommand\labelitemii{$\vcenter{\hbox{\tiny$\bullet$}}$}

\newcommand{\resumeSubHeadingListStart}{\begin{itemize}[leftmargin=0.15in, label={}]}
\newcommand{\resumeSubHeadingListEnd}{\end{itemize}}
\newcommand{\resumeItemListStart}{\begin{itemize}}
\newcommand{\resumeItemListEnd}{\end{itemize}\vspace{-5pt}}

%-------------------------------------------
%%%%%%  RESUME STARTS HERE  %%%%%%%%%%%%%%%%%%%%%%%%%%%%


\begin{document}

%----------HEADING----------

\begin{center}
    \textbf{\Huge \scshape Souvik Kar Mahapatra} \\ \vspace{1pt}
    \href{mailto:souvikat001@gmail.com}{\underline{souvikat001@gmail.com}} $|$ 
    \href{https://twitter.com/souvikinator}{\underline{twitter}} $|$ 
    \href{https://linkedin.com/in/souvik-kar-mahapatra}{\underline{linkedin}} $|$
    \href{https://github.com/souvikinator}{\underline{Github}}
    
\end{center}


%-----------EXPERIENCE-----------
\section{Experience}
  \resumeSubHeadingListStart
% -----------Multiple Positions Heading-----------
%    \resumeSubSubheading
%     {Software Engineer I}{Oct 2014 - Sep 2016}
%     \resumeItemListStart
%        \resumeItem{Apache Beam}
%          {Apache Beam is a unified model for defining both batch and streaming data-parallel processing pipelines}
%     \resumeItemListEnd
%    \resumeSubHeadingListEnd
%-------------------------------------------
    \resumeSubheading
      {Tyup}{Remote}
      {Lead Back-end Engineer}{June 2021 -- Present}
      \resumeItemListStart
        \resumeItem{Developerd whole back-end system from scratch. Token based authentication(JWT), feed based on institution visibility, email OTP system using redis, Reset and forgot password, Protection against vulnerabilities like XSS, HPP and IDOR.}
        \resumeItem{Integrated nginx as reverse proxy }
        \resumeItem{Dockerized the whole application using docker-compose}
        \resumeItem{Integrated AWS S3 with in-server authentication to prevent unauthorised access to tyup resources.}
        \resumeItem{Migrated all the data from old MongoDB instance to New MongoDB instance.}
        \resumeItem{Serving 500+ active users and survived various injection attacks easily.}
        
    
      \resumeItemListEnd
    \resumeSubheading
      {MnyRate}{Remote}
      {On Contract}{April 2021 -- May 2021}
      \resumeItemListStart
        \resumeItem{Developed session based authentication, API endpoints for various operations on post on user profile}
        \resumeItem{Developed admin dashboard.}
     \resumeItemListEnd
  \resumeSubHeadingListEnd


%-----------PROJECTS-----------
\section{Projects}
    \resumeSubHeadingListStart
      \resumeProjectHeading
          {\textbf{\href{https://github.com/souvikinator/lsx}{lsx (ls-Xtended):  A command line utility lets you navigate through terminal like a pro }} {[ Go, Shell ]}}{October 2021}
          \resumeItemListStart
           \resumeItem{\textbf{Problem:} It's a pain to cd and ls multiple times to reach desired directory in terminal and makes the whole navigation process slow }
           \resumeItem{\textbf{Solution:} lsx allows smooth navigate and search directories on the go with just one command. Create alias for desired directory. Ranks most visited directories for better accessibility.}
           \resumeItem{ 100+ stars on github with active contribution from users }
          \resumeItemListEnd
          
          \resumeProjectHeading
          {\textbf{\href{https://github.com/souvikinator/gofuzz}{gofuzz: web url fuzzer }} {[ Go ]}}{March 2021}
          \resumeItemListStart
           \resumeItem{\textbf{Problem:} While solving CTFs it was annoying to test URLs for IDOR, HPP and other injections attacks manually and check their status codes. }
           \resumeItem{\textbf{Solution:} Test for numbers within a certain range, ascii characters and inputs from files. Uses golang's concurrency for fast fuzzing. Categorises inputs as per the status code. Export result to various file formats for further automation.}
           \resumeItem{ 30+ stars on github }
          \resumeItemListEnd
          
         \resumeProjectHeading
          {\textbf{\href{https://github.com/souvikinator/synko}{synko: sync config files across devices }} {[ python, shell ]}}{March 2021}
          \resumeItemListStart
           \resumeItem{\textbf{Problem:} Difficult to sync application settings and config files across multiple multiplatform devices }
           \resumeItem{\textbf{Solution:} Sync application settings and configuration files across multiple devices (linux and macos). Works with dropbox and other cloud storage. User gets the freedom to add path to the config and setting files. Freedom to sync specific set of files across specific set of devices. }
          \resumeItemListEnd
          
          \resumeProjectHeading
          {\textbf{\href{https://github.com/souvikinator/unwee}{unwee: URL unshortner }} {[ Go ]}}{December 2020}
          \resumeItemListStart
           \resumeItem{\textbf{Problem:} URL shortening services are uses by attackers to redirect user to malicious sites. Automation in Web-scraping and related, short URLs are often encountered but needs to be unshortned with some other online service. }
           \resumeItem{\textbf{Solution:} Unwee can be used to unshorten URLs knowing before hand where it leads to. Can be used as an addon with other automation tools for bulk unshortning of URLs. Also classifies URLs based on status codes.}
          \resumeItemListEnd  
          
          \resumeProjectHeading
          {\textbf{\href{https://github.com/souvikinator/node-eyeson}{eyeson: minimal file watching action trigger CLI}} {[ NodeJS ]}}{October 2020}
          \resumeItemListStart
          \resumeItem{\textbf{Problem:} It is tedious to make changes to program, compile \& run multiple time manually }
           \resumeItem{\textbf{Solution:} eyeson watches file for any changes which triggers some command provided by user. Can watch multiple files. Works with glob patterns. }
           \resumeItem{ used by 100+ downloads on npm }
          \resumeItemListEnd
        
        \newpage
        
          \resumeProjectHeading
          {\textbf{\href{https://github.com/souvikinator/web-SS}{web-SS: Screenshots of the websites in bulk}} {[ NodeJS, Express, EJS ]}}{November 2020}
          \resumeItemListStart
           \resumeItem{\textbf{Problem:} Takings screenshots of multiple pages of website for showcase purpose is a lot of work to do.}
           \resumeItem{\textbf{Solution:} web-SS takes multiple URLs as input, generates screenshot in bulk automating the screenshot process. Support for viewport and long screenshot with custom dimentions. }
          \resumeItemListEnd  
         
         \resumeProjectHeading
          {\textbf{\href{https://github.com/souvikinator/asker}{asker: A collection of interactive command line user interfaces for C++}} {[ C++, Makefile ]}}{September 2021}
          \resumeItemListStart
           \resumeItem{\textbf{Problem:} Inspired by \href{https://www.npmjs.com/package/inquirer}{inquirer}, no such library for C++}
           \resumeItem{\textbf{Solution:} Asker provides command line interface like confirm prompt, dropdown, multi-select dropdown, password input, masked input, normal input with and without validation. }
          \resumeItemListEnd  
          
          \resumeProjectHeading
          {\textbf{\href{https://github.com/souvikinator/gh-retrieve}{gh-retrieve: Download specific directory or sub-directory in a public GitHub repository}} {[ NodeJS ]}}{April 2021}
          \resumeItemListStart
           \resumeItem{\textbf{Problem:} Github won't let you clone or download any specific directory or sub-directory in a repo.}
           \resumeItem{\textbf{Solution:} Nodejs module to download any specific directory or sub-directory in a public GitHub repository. Recursive download, uses github API, doesn't maintain any repo workflow. Sparse download, uses git installed in system, maintains git workflow of the target repository. }
           \resumeItem{150+ downloads on npm.}
          \resumeItemListEnd  
    \resumeSubHeadingListEnd

%
%-----------PROGRAMMING SKILLS-----------
\section{Technical Skills}
 \begin{itemize}[leftmargin=0.15in, label={}]
    \small{\item{
     \textbf{Languages(Proficient)}{: JavaScript, C++, Go, Bash, Python } \\
      \textbf{Languages(Familiar)}{: TypeScript, C, Rust } \\
     \textbf{Frameworks}{: NextJS, PostgreSQL, Redis, SQL, gRPC, React, NodeJS  } \\
     \textbf{Developer Tools}{: AWS(S3, EC2) , Docker, Nginx, Postman, Github Actions } \\
    }}
 \end{itemize}
 
%-----------EDUCATION-----------
\section{Education}
  \resumeSubHeadingListStart
    \resumeSubheading
      {Kalinga Institute of Industrial Technology}{Bhuvaneshwar, Odissa}
      {Sophomore year, B.Tech in Information Technology}{September 2020 -- Expected July. 2024}
  \resumeSubHeadingListEnd
\end{document}
